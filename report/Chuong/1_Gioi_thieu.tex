\documentclass[../DoAn.tex]{subfiles}
\begin{document}

\section{Đặt vấn đề}
\label{section:1.1}

Trong những năm gần đây, các nền tảng giao đồ ăn trực tuyến đã trở thành một kênh tiêu dùng phổ biến nhờ khả năng đáp ứng nhanh, tiện lợi và phù hợp với nhịp sống đô thị. Với nhóm sinh viên và nhân viên văn phòng, nhu cầu đặt đồ ăn theo nhóm trong các khung giờ giới hạn (đặc biệt là bữa trưa) diễn ra thường xuyên, kéo theo yêu cầu về thao tác đặt món nhanh, minh bạch chi phí và phối hợp nhóm thuận tiện.

Mặc dù các nền tảng phổ biến đã cung cấp nhiều chức năng cốt lõi của một hệ thống giao đồ ăn, trải nghiệm đặt hàng theo nhóm vẫn còn tồn tại hạn chế đáng kể ở khâu thanh toán. Trong nhiều kịch bản thực tế, một người thường phải đứng ra thanh toán toàn bộ đơn hàng, sau đó tự thu lại tiền từ các thành viên còn lại. Quy trình này làm tăng số bước thao tác ngoài ứng dụng, tốn thời gian, và dễ phát sinh sai lệch hoặc chậm trễ trong việc hoàn trả chi phí, đặc biệt trong môi trường tập thể.

Từ bối cảnh đó, đồ án lựa chọn giải quyết bài toán đặt hàng nhóm theo hướng giảm thiểu trở ngại trong phối hợp và thanh toán, đồng thời vẫn đảm bảo đầy đủ hành trình đặt món của người dùng và khả năng vận hành của phía nhà hàng. Trọng tâm của dự án là tính năng \gls{teamcart} (Giỏ hàng nhóm), cho phép nhiều người cùng tham gia chọn món trong một giỏ hàng, theo dõi trạng thái cập nhật tức thời, và hỗ trợ cơ chế thanh toán theo từng thành viên theo định hướng tách bạch trách nhiệm chi trả.

\section{Mục tiêu và phạm vi đề tài}
\label{section:1.2}

Từ vấn đề đã nêu, mục tiêu của dự án \gls{yummyzoom} là xây dựng một hệ thống giao đồ ăn ở mức sản phẩm khả dụng tối thiểu (\gls{mvp}), trong đó nổi bật là trải nghiệm đặt hàng nhóm \gls{teamcart}, hỗ trợ nhiều người cùng chọn món và tách bạch việc thanh toán theo từng thành viên. Song song với chức năng trọng tâm, hệ thống cần đáp ứng các chức năng cốt lõi của một nền tảng giao đồ ăn, đảm bảo người dùng có thể hoàn thành hành trình đặt món, phía nhà hàng có thể vận hành đơn, và nền tảng có cơ chế quản trị phù hợp.

Về phạm vi chức năng, hệ thống được xây dựng theo ba nhóm vai trò chính. Đối với khách hàng, hệ thống hỗ trợ duyệt nhà hàng và thực đơn, tùy chỉnh món ăn, đặt hàng cá nhân, tham gia \gls{teamcart}, theo dõi trạng thái đơn hàng và đánh giá sau khi sử dụng dịch vụ. Đối với nhà hàng, hệ thống hỗ trợ quản lý hồ sơ và thực đơn, cập nhật trạng thái món, tiếp nhận và xử lý đơn hàng theo vòng đời, và theo dõi các thông tin vận hành liên quan. Đối với quản trị viên, hệ thống cung cấp các chức năng quản lý nền tảng như duyệt đăng ký nhà hàng, giám sát nội dung và quản trị dữ liệu phục vụ vận hành.

Để đảm bảo tính khả thi trong khuôn khổ đồ án tốt nghiệp, phạm vi triển khai được giới hạn ở mức phù hợp. Quy trình giao nhận được mô phỏng thông qua cập nhật trạng thái đơn hàng thay vì phát triển một phân hệ tài xế với các nghiệp vụ định vị, điều phối và theo dõi GPS thời gian thực. Bên cạnh đó, hệ thống chưa triển khai các tính năng nâng cao như gợi ý món ăn, tìm kiếm ứng dụng học máy theo hướng ngữ nghĩa, tích hợp cổng thanh toán ở mức sẵn sàng sản xuất, cũng như module tài chính chuẩn cho phía nhà hàng (ví dụ sổ cái doanh thu, đối soát theo kỳ, và luồng rút tiền).

\section{Định hướng giải pháp}
\label{section:1.3}

Để hiện thực hóa mục tiêu đề tài, đồ án lựa chọn hướng tiếp cận ưu tiên tính mô-đun, khả năng bảo trì và khả năng mở rộng trong triển khai. Ở mức kiến trúc, hệ thống áp dụng \gls{cleanarchitecture} kết hợp Domain-Driven Design (\gls{ddd}) nhằm tách biệt rõ logic nghiệp vụ khỏi các chi tiết hạ tầng, qua đó giảm phụ thuộc giữa các lớp, hỗ trợ kiểm thử và cho phép mở rộng chức năng theo từng miền nghiệp vụ. Các quy tắc quan trọng của \gls{teamcart} được mô hình hóa tập trung trong lớp miền (Domain layer), trong khi lớp ứng dụng (Application layer) điều phối các ca sử dụng theo định hướng \gls{cqrs}.

Ở mức công nghệ, hệ thống backend được xây dựng trên \gls{dotnet} 9 và \gls{aspnetcore}, triển khai \gls{api} phục vụ ứng dụng khách hàng và cổng quản trị. \gls{postgresql} được sử dụng làm hệ quản trị cơ sở dữ liệu chính để lưu trữ dữ liệu bền vững, \gls{redis} được sử dụng cho các nhu cầu lưu trữ trạng thái nhanh và hỗ trợ các luồng cộng tác, trong khi \gls{signalr} đảm nhiệm kênh giao tiếp thời gian thực (real-time) để các thành viên trong \gls{teamcart} có thể nhận cập nhật tức thời. Ở phía giao diện, ứng dụng khách hàng được phát triển bằng \gls{flutter} để đảm bảo trải nghiệm nhất quán trên iOS và Android, còn cổng quản trị nhà hàng và quản trị viên được xây dựng theo mô hình Single Page Application (SPA) sử dụng \gls{angular} và PrimeNG. Các lựa chọn công nghệ và cách áp dụng cụ thể sẽ được trình bày chi tiết trong các chương tiếp theo.

\section{Bố cục đồ án}
\label{section:1.4}

Phần còn lại của báo cáo được tổ chức như sau. Chương 2 trình bày quá trình khảo sát và phân tích yêu cầu của hệ thống \gls{yummyzoom}, bao gồm bối cảnh bài toán, yêu cầu chức năng và phi chức năng làm cơ sở cho thiết kế và triển khai. Chương 3 giới thiệu các công nghệ và giải pháp kỹ thuật được lựa chọn để hiện thực hệ thống theo hướng \gls{cleanarchitecture} và \gls{ddd}, đồng thời trình bày các thành phần hạ tầng phục vụ lưu trữ dữ liệu và cộng tác thời gian thực. Chương 4 mô tả thiết kế, triển khai và đánh giá hệ thống, bao gồm kiến trúc tổng thể, thiết kế cơ sở dữ liệu, hiện thực các chức năng chính và kiểm thử. Chương 5 tổng hợp các giải pháp đóng góp nổi bật, đánh giá sản phẩm trong bối cảnh thị trường, nêu các hạn chế còn tồn tại và đề xuất hướng phát triển cho dự án.

\iffalse

%%%%%%%%%%%%%%%%%%%%%%%%%%%%%%%%%%%%%%%%%%%%%%%%%%%%%%%%%%%%%%%%%%%%%%%%%%%
%%% PHẦN HƯỚNG DẪN TỪ MẪU BÁO CÁO - KHÔNG XÓA %%%
%%%%%%%%%%%%%%%%%%%%%%%%%%%%%%%%%%%%%%%%%%%%%%%%%%%%%%%%%%%%%%%%%%%%%%%%%%%
% \begin{center}
% \textbf{PHẦN HƯỚNG DẪN TỪ MẪU BÁO CÁO}
% \end{center}

% Lưu ý: \textbf{Trước khi viết ĐATN, sinh viên cần đọc kỹ hướng dẫn và quy định chi tiết} về cách viết ĐATN trong Phụ lục A. Sinh viên tuân theo mẫu tài liệu này để viết báo cáo đồ án tốt nghiệp, vì tài liệu này đã được căn chỉnh, chỉnh sửa theo đúng chuẩn báo cáo kỹ thuật đồ án tốt nghiệp (ISO 7144:1986). Sinh viên viết trực tiếp vào file này, chỉ chỉnh sửa nội dung, và không viết trên file mới.

% \textbf{Khi đóng quyển ĐATN}, sinh viên cần lưu ý tuân thủ hướng dẫn ở phụ lục A.9

% \textbf{SV cần đặc biệt lưu ý cách hành văn}. Mỗi đoạn văn không được quá dài và cần có ý tứ rõ ràng, bao gồm duy nhất một ý chính và các ý phân tích bổ trợ để làm rõ hơn ý chính. Các câu văn trong đoạn phải đầy đủ chủ ngữ vị ngữ, cùng hướng đến chủ đề chung. Câu sau phải liên kết với câu trước, đoạn sau liên kết với đoạn trước. Trong văn phong khoa học, sinh viên không được dùng từ trong văn nói, không dùng các từ phóng đại, thái quá, các từ thiếu khách quan, thiên về cảm xúc, về quan điểm cá nhân như “tuyệt vời”, “cực hay”, “cực kỳ hữu ích”, v.v. Các câu văn cần được tối ưu hóa, đảm bảo rất khó để thể thêm hoặc bớt đi được dù chỉ một từ. Cách diễn đạt cần ngắn gọn, súc tích, không dài dòng.

% Mẫu ĐATN này được thiết kế phù hợp nhất với đa số các đề tài xây dựng phần mềm ứng dụng. Với các dạng đề tài khác (giải pháp, nghiên cứu, phần mềm đặc thù, v.v.), sinh viên dựa trên cấu trúc và hướng dẫn của báo cáo này để đề xuất và trao đổi với giáo viên hướng dẫn để thiết kế khung báo cáo đồ án cho phù hợp. Sinh viên lưu ý \textbf{trong mọi trường hợp, SV luôn phải sử dụng định dạng báo cáo này, và phải đọc kỹ toàn bộ các hướng dẫn từ đầu tới cuối.} Các hướng dẫn không chỉ áp dụng riêng cho đề tài ứng dụng, mà còn phù hợp với các dạng đề tài khác. Ngoài ra, trong mẫu ĐATN này đã được tích hợp một số hướng dẫn dành riêng cho đề tài nghiên cứu.

% Chương 1 có độ dài từ 3 đến 6 trang với các nội dung sau đây

% \subsection*{Hướng dẫn viết phần 1.1 - Đặt vấn đề}
% Khi đặt vấn đề, sinh viên cần làm nổi bật mức độ cấp thiết, tầm quan trọng và/hoặc quy mô của bài toán của mình.

% Gợi ý cách trình bày cho sinh viên: Xuất phát từ tình hình thực tế gì, dẫn đến vấn đề hoặc bài toán gì. Vấn đề hoặc bài toán đó, nếu được giải quyết, đem lại lợi ích gì, cho những ai, còn có thể được áp dụng vào các lĩnh vực khác nữa không. Sinh viên cần lưu ý phần này chỉ trình bày vấn đề, tuyệt đối không trình bày giải pháp.

% \subsection*{Hướng dẫn viết phần 1.2 - Mục tiêu và phạm vi đề tài}
% Sinh viên trước tiên cần trình bày tổng quan các kết quả của các nghiên cứu hiện nay cho bài toán giới thiệu ở phần 1.1 (đối với đề tài nghiên cứu), hoặc về các sản phẩm hiện tại/về nhu cầu của người dùng (đối với đề tài ứng dụng). Tiếp đến, sinh viên tiến hành so sánh và đánh giá tổng quan các sản phẩm/nghiên cứu này.

% Dựa trên các phân tích và đánh giá ở trên, sinh viên khái quát lại các hạn chế hiện tại đang gặp phải. Trên cơ sở đó, sinh viên sẽ hướng tới giải quyết vấn đề cụ thể gì, khắc phục hạn chế gì, phát triển phần mềm \textbf{có các chức năng chính gì}, tạo nên đột phá gì, v.v.

% Trong phần này, sinh viên lưu ý chỉ trình bày tổng quan, không đi vào chi tiết của vấn đề hoặc giải pháp. Nội dung chi tiết sẽ được trình bày trong các chương tiếp theo, đặc biệt là trong Chương 5.

% \subsection*{Hướng dẫn viết phần 1.3 - Định hướng giải pháp}
% Từ việc xác định rõ nhiệm vụ cần giải quyết ở phần 1.2, sinh viên đề xuất định hướng giải pháp của mình theo trình tự sau: (i) Sinh viên trước tiên trình bày sẽ giải quyết vấn đề theo định hướng, phương pháp, thuật toán, kỹ thuật, hay công nghệ nào; Tiếp theo, (ii) sinh viên mô tả ngắn gọn giải pháp của mình là gì (khi đi theo định hướng/phương pháp nêu trên); và sau cùng, (iii) sinh viên trình bày đóng góp chính của đồ án là gì, kết quả đạt được là gì.

% Sinh viên lưu ý không giải thích hoặc phân tích chi tiết công nghệ/thuật toán trong phần này. Sinh viên chỉ cần nêu tên định hướng công nghệ/thuật toán, mô tả ngắn gọn trong một đến hai câu và giải thích nhanh lý do lựa chọn.

% \subsection*{Hướng dẫn viết phần 1.4 - Bố cục đồ án}
% Phần còn lại của báo cáo đồ án tốt nghiệp này được tổ chức như sau. 

% Chương 2 trình bày về v.v. 

% Trong Chương 3, em/tôi giới thiệu về v.v.

% \textbf{Chú ý:} Sinh viên cần viết mô tả thành đoạn văn đầy đủ về nội dung chương. Tuyệt đối không viết ý hay gạch đầu dòng. Chương 1 không cần mô tả trong phần này. 

% Ví dụ tham khảo mô tả chương trong phần bố cục đồ án tốt nghiệp: Chương *** trình bày đóng góp chính của đồ án, đó là một nền tảng ABC cho phép khai phá và tích hợp nhiều nguồn dữ liệu, trong đó mỗi nguồn dữ liệu lại có định dạng đặc thù riêng. Nền tảng ABC được phát triển dựa trên khái niệm DEF, là các module ngữ nghĩa trợ giúp người dùng tìm kiếm, tích hợp và hiển thị trực quan dữ liệu theo mô hình cộng tác và mô hình phân tán.

% \textbf{Chú ý:} Trong phần nội dung chính, mỗi chương của đồ án nên có phần Tổng quan và Kết chương. Hai phần này đều có định dạng văn bản “Normal”, sinh viên không cần tạo định dạng riêng, ví dụ như không in đậm/in nghiêng, không đóng khung, v.v. 

% Trong phần Tổng quan của chương N, sinh viên nên có sự liên kết với chương N-1 rồi trình bày sơ qua lý do có mặt của chương N và sự cần thiết của chương này trong đồ án. Sau đó giới thiệu những vấn đề sẽ trình bày trong chương này là gì, trong các đề mục lớn nào.

% Ví dụ về phần Tổng quan: Chương 3 đã thảo luận về nguồn gốc ra đời, cơ sở lý thuyết và các nhiệm vụ chính của bài toán tích hợp dữ liệu. Chương 4 này sẽ trình bày chi tiết các công cụ tích hợp dữ liệu theo hướng tiếp cận “mashup”. Với mục đích và phạm vi của đề tài, sáu nhóm công cụ tích hợp dữ liệu chính được trình bày bao gồm: (i) nhóm công cụ ABC trong phần 4.1, (ii) nhóm công cụ DEF trong phần 4.2, nhóm công cụ GHK trong phần 4.3, v.v.

% Trong phần Kết chương, sinh viên đưa ra một số kết luận quan trọng của chương. Những vấn đề mở ra trong Tổng quan cần được tóm tắt lại nội dung và cách giải quyết/thực hiện như thế nào. Sinh viên lưu ý không viết Kết chương giống hệt Tổng quan. Sau khi đọc phần Kết chương, người đọc sẽ nắm được sơ bộ nội dung và giải pháp cho các vấn đề đã trình bày trong chương. Trong Kết chương, Sinh viên nên có thêm câu liên kết tới chương tiếp theo.

% Ví dụ về phần Kết chương: Chương này đã phân tích chi tiết sáu nhóm công cụ tích hợp dữ liệu. Nhóm công cụ ABC và DEF thích hợp với những bài toán tích hợp dữ liệu phạm vi nhỏ. Trong khi đó, nhóm công cụ GHK lại chứng tỏ thế mạnh của mình với những bài toán cần độ chính xác cao, v.v. Từ kết quả nghiên cứu và phân tích về sáu nhóm công cụ tích hợp dữ liệu này, tôi đã thực hiện phát triển phần mềm tự động bóc tách và tích hợp dữ liệu sử dụng nhóm công cụ GHK. Phần này được trình bày trong chương tiếp theo – Chương 5.

\fi
\end{document}
