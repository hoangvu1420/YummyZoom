\documentclass[../DoAn.tex]{subfiles}
\begin{document}

\section{Đặc tả chức năng}
\label{section:2.3}

% Sinh viên lựa chọn từ 4 đến 7 use case quan trọng nhất của đồ án để đặc tả chi tiết. Mỗi đặc tả bao gồm ít nhất các thông tin sau: (i) Tên use case, (ii) Luồng sự kiện (chính và phát sinh), (iii) Tiền điều kiện, và (iv) Hậu điều kiện. Sinh viên chỉ vẽ bổ sung biểu đồ hoạt động khi đặc tả use case phức tạp.

% \clearpage
\subsection{Đặc tả Use Case: Đăng ký nhà hàng}
\begin{longtable}{|p{2.5cm}|p{1.5cm}|p{3cm}|p{5.5cm}|}
    \hline
    \textbf{Mã use case} & UC-007 & \textbf{Tên use case} & Đăng ký nhà hàng \\ \hline
    \textbf{Tác nhân} & \multicolumn{3}{p{11.5cm}|}{Khách hàng (Đối tác tiềm năng)} \\ \hline
    \textbf{Mục đích sử dụng} & \multicolumn{3}{p{11.5cm}|}{Cho phép người dùng gửi yêu cầu đăng ký mở nhà hàng mới trên hệ thống.} \\ \hline
    \textbf{Sự kiện kích hoạt} & \multicolumn{3}{p{11.5cm}|}{Người dùng nhấn nút "Đăng ký quán" trên giao diện.} \\ \hline
    \textbf{Tiền điều kiện} & \multicolumn{3}{p{11.5cm}|}{Người dùng đã đăng nhập và tài khoản đang hoạt động.} \\ \hline
    \endfirsthead
    
    \multicolumn{4}{r}{\textit{(Tiếp theo từ trang trước)}} \\
    \hline
    \endhead
    
    \hline
    \multicolumn{4}{r}{\textit{(Tiếp tục trang sau)}} \\
    \endfoot
    
    \caption{Đặc tả Use Case Đăng ký nhà hàng} \label{tab:uc_007} \\
    \endlastfoot

    \multirow{4}{2.5cm}{\textbf{Luồng sự kiện chính}} & \textbf{STT} & \textbf{Thực hiện bởi} & \textbf{Hành động} \\ \cline{2-4} 
     & 1 & Khách hàng & Nhập thông tin nhà hàng (Tên, địa chỉ, liên hệ, giờ mở cửa...). \\ \cline{2-4} 
     & 2 & Hệ thống & Kiểm tra tính hợp lệ của dữ liệu nhập vào. \\ \cline{2-4} 
     & 3 & Hệ thống & Tạo hồ sơ đăng ký với trạng thái "Đã nộp" (Submitted). \\ \cline{2-4} 
     & 4 & Hệ thống & Thông báo đăng ký thành công và chờ duyệt. \\ \hline
    \multirow{2}{2.5cm}{\textbf{Luồng sự kiện thay thế}} & \textbf{STT} & \textbf{Thực hiện bởi} & \textbf{Hành động} \\ \cline{2-4} 
     & 2a & Hệ thống & Hiển thị thông báo lỗi nếu dữ liệu thiếu hoặc sai định dạng. \\ \hline
    \textbf{Hậu điều kiện} & \multicolumn{3}{p{11.5cm}|}{Hồ sơ đăng ký được lưu vào hệ thống, chờ Admin phê duyệt.} \\ \hline
\end{longtable}

Dữ liệu đầu vào cho use case Đăng ký nhà hàng:

\begin{longtable}{|c|p{3cm}|p{4cm}|c|p{3cm}|}
    \hline
    \textbf{STT} & \textbf{Trường dữ liệu} & \textbf{Mô tả} & \textbf{Bắt buộc?} & \textbf{Điều kiện hợp lệ} \\ \hline
    \endfirsthead
    
    \multicolumn{5}{r}{\textit{(Tiếp theo từ trang trước)}} \\
    \hline
    \textbf{STT} & \textbf{Trường dữ liệu} & \textbf{Mô tả} & \textbf{Bắt buộc?} & \textbf{Điều kiện hợp lệ} \\ \hline
    \endhead
    
    \hline
    \multicolumn{5}{r}{\textit{(Tiếp tục trang sau)}} \\
    \endfoot
    
    \caption{Dữ liệu đầu vào Đăng ký nhà hàng} \label{tab:input_uc_007} \\
    \endlastfoot

    1 & Name & Tên hiển thị của quán & Có & Tối đa 100 ký tự \\ \hline
    2 & Description & Giới thiệu ngắn gọn & Có & Tối đa 500 ký tự \\ \hline
    3 & CuisineType & Ví dụ: Cơm, Phở, Trà sữa & Có & Tối đa 50 ký tự \\ \hline
    4 & Street & Số nhà, tên đường & Có & Tối đa 200 ký tự \\ \hline
    5 & City & Tên thành phố & Có & Tối đa 100 ký tự \\ \hline
    6 & State & Tên tỉnh hoặc bang & Có & Tối đa 100 ký tự \\ \hline
    7 & ZipCode & Zip Code & Có & Tối đa 20 ký tự \\ \hline
    8 & Country & Tên quốc gia & Có & Tối đa 100 ký tự \\ \hline
    9 & PhoneNumber & SĐT liên hệ của quán & Có & Tối đa 30 ký tự \\ \hline
    10 & Email & Email liên hệ & Có & Đúng định dạng Email \\ \hline
    11 & BusinessHours & Khung giờ hoạt động & Có & Tối đa 200 ký tự \\ \hline
    12 & LogoUrl & Đường dẫn ảnh đại diện & Không & URL hợp lệ \\ \hline
    13 & Latitude & Tọa độ địa lý & Không & -90 đến 90 \\ \hline
    14 & Longitude & Tọa độ địa lý & Không & -180 đến 180 \\ \hline
\end{longtable}


\clearpage
\subsection{Đặc tả Use Case: Duyệt đăng ký nhà hàng}
\begin{longtable}{|p{2.5cm}|p{1.5cm}|p{3cm}|p{5.5cm}|}
    \hline
    \textbf{Mã use case} & UC-012 & \textbf{Tên use case} & Duyệt đăng ký nhà hàng \\ \hline
    \textbf{Tác nhân} & \multicolumn{3}{p{11.5cm}|}{Quản trị viên (Admin)} \\ \hline
    \textbf{Mục đích sử dụng} & \multicolumn{3}{p{11.5cm}|}{Cho phép Admin xem xét và phê duyệt yêu cầu đăng ký nhà hàng, đồng thời khởi tạo nhà hàng mới trên hệ thống.} \\ \hline
    \textbf{Sự kiện kích hoạt} & \multicolumn{3}{p{11.5cm}|}{Admin nhấn nút "Duyệt" (Approve) trên chi tiết hồ sơ đăng ký.} \\ \hline
    \textbf{Tiền điều kiện} & \multicolumn{3}{p{11.5cm}|}{Admin đã đăng nhập. Hồ sơ đăng ký tồn tại và đang ở trạng thái chờ duyệt (Submitted/UnderReview).} \\ \hline
    \endfirsthead
    
    \multicolumn{4}{r}{\textit{(Tiếp theo từ trang trước)}} \\
    \hline
    \endhead
    
    \hline
    \multicolumn{4}{r}{\textit{(Tiếp tục trang sau)}} \\
    \endfoot
    
    \caption{Đặc tả Use Case Duyệt đăng ký nhà hàng} \label{tab:uc_012} \\
    \endlastfoot

    \multirow{5}{2.5cm}{\textbf{Luồng sự kiện chính}} & \textbf{STT} & \textbf{Thực hiện bởi} & \textbf{Hành động} \\ \cline{2-4} 
     & 1 & Admin & Nhập ghi chú (tùy chọn) và xác nhận duyệt. \\ \cline{2-4} 
     & 2 & Hệ thống & Tạo mới thực thể Nhà hàng (Restaurant) từ thông tin đăng ký. \\ \cline{2-4} 
     & 3 & Hệ thống & Gán quyền Chủ sở hữu (Owner) cho tài khoản người đăng ký. \\ \cline{2-4} 
     & 4 & Hệ thống & Cập nhật trạng thái hồ sơ thành "Đã duyệt" (Approved). \\ \cline{2-4} 
     & 5 & Hệ thống & Gửi thông báo thành công cho Admin và người đăng ký. \\ \hline
    \multirow{2}{2.5cm}{\textbf{Luồng sự kiện thay thế}} & \textbf{STT} & \textbf{Thực hiện bởi} & \textbf{Hành động} \\ \cline{2-4} 
     & 2a & Hệ thống & Báo lỗi nếu quá trình tạo nhà hàng hoặc gán quyền thất bại. \\ \hline
    \textbf{Hậu điều kiện} & \multicolumn{3}{p{11.5cm}|}{Nhà hàng mới được tạo và hiển thị trên hệ thống. Người đăng ký có quyền quản lý nhà hàng đó.} \\ \hline
\end{longtable}

Dữ liệu đầu vào cho use case Duyệt đăng ký nhà hàng:

\begin{longtable}{|c|p{3cm}|p{4cm}|c|p{3cm}|}
    \hline
    \textbf{STT} & \textbf{Trường dữ liệu} & \textbf{Mô tả} & \textbf{Bắt buộc?} & \textbf{Điều kiện hợp lệ} \\ \hline
    \endfirsthead
    
    \multicolumn{5}{c}{\textit{(Tiếp theo từ trang trước)}} \\
    \hline
    \textbf{STT} & \textbf{Trường dữ liệu} & \textbf{Mô tả} & \textbf{Bắt buộc?} & \textbf{Điều kiện hợp lệ} \\ \hline
    \endhead
    
    \hline
    \multicolumn{5}{r}{\textit{(Tiếp tục trang sau)}} \\
    \endfoot
    
    \caption{Dữ liệu đầu vào Duyệt đăng ký nhà hàng} \label{tab:input_uc_012} \\
    \endlastfoot

    1 & RegistrationId & ID của hồ sơ đăng ký & Có & GUID hợp lệ, tồn tại \\ \hline
    2 & Note & Ghi chú nội bộ của Admin & Không & Tối đa 500 ký tự \\ \hline
\end{longtable}


\clearpage
\subsection{Đặc tả Use Case: Quản lý thực đơn}
\begin{longtable}{|p{2.5cm}|p{1.5cm}|p{3cm}|p{5.5cm}|}
    \hline
    \textbf{Mã use case} & UC-008 & \textbf{Tên use case} & Quản lý thực đơn \\ \hline
    \textbf{Tác nhân} & \multicolumn{3}{p{11.5cm}|}{Chủ nhà hàng (Restaurant Owner/Staff)} \\ \hline
    \textbf{Mục đích sử dụng} & \multicolumn{3}{p{11.5cm}|}{Cho phép nhà hàng tạo, cập nhật, xóa danh mục món ăn, món ăn và nhóm tùy chọn (size, topping) để xây dựng thực đơn hoàn chỉnh.} \\ \hline
    \textbf{Sự kiện kích hoạt} & \multicolumn{3}{p{11.5cm}|}{Chủ nhà hàng truy cập trang quản lý thực đơn và thực hiện các thao tác CRUD.} \\ \hline
    \textbf{Tiền điều kiện} & \multicolumn{3}{p{11.5cm}|}{Người dùng đã đăng nhập và có quyền quản lý nhà hàng (Owner hoặc Staff). Nhà hàng đã được duyệt và tồn tại trong hệ thống.} \\ \hline
    \endfirsthead
    
    \multicolumn{4}{r}{\textit{(Tiếp theo từ trang trước)}} \\
    \hline
    \endhead
    
    \hline
    \multicolumn{4}{r}{\textit{(Tiếp tục trang sau)}} \\
    \endfoot
    
    \caption{Đặc tả Use Case Quản lý thực đơn} \label{tab:uc_008} \\
    \endlastfoot

    \multirow{9}{2.5cm}{\textbf{Luồng sự kiện chính}} & \textbf{STT} & \textbf{Thực hiện bởi} & \textbf{Hành động} \\ \cline{2-4} 
     & 1 & Chủ nhà hàng & Chọn chức năng quản lý danh mục/món ăn/nhóm tùy chọn. \\ \cline{2-4} 
     & 2 & Chủ nhà hàng & Nhập thông tin chi tiết (tên, mô tả, giá, hình ảnh...). \\ \cline{2-4} 
     & 3 & Hệ thống & Kiểm tra tính hợp lệ của dữ liệu đầu vào (không rỗng, giá > 0, định dạng URL...). \\ \cline{2-4} 
     & 4 & Hệ thống & Kiểm tra quyền sở hữu (danh mục/món phải thuộc nhà hàng đang quản lý). \\ \cline{2-4} 
     & 5 & Hệ thống & Tạo/cập nhật thực thể trong cơ sở dữ liệu (MenuItem, MenuCategory). \\ \cline{2-4} 
     & 6 & Hệ thống & Phát sự kiện domain (MenuItemCreated, MenuItemUpdated, MenuItemDeleted...). \\ \cline{2-4} 
     & 7 & Hệ thống & Cập nhật read model FullMenuView để đồng bộ dữ liệu hiển thị. \\ \cline{2-4} 
     & 8 & Hệ thống & Thông báo thành công và hiển thị danh sách cập nhật. \\ \hline
    \multirow{4}{2.5cm}{\textbf{Luồng sự kiện thay thế}} & \textbf{STT} & \textbf{Thực hiện bởi} & \textbf{Hành động} \\ \cline{2-4} 
     & 3a & Hệ thống & Hiển thị lỗi nếu dữ liệu không hợp lệ (tên rỗng, giá âm, URL sai định dạng). \\ \cline{2-4} 
     & 4a & Hệ thống & Từ chối thao tác nếu món ăn/danh mục không thuộc nhà hàng (ForbiddenAccessException). \\ \cline{2-4} 
     & 5a & Hệ thống & Báo lỗi nếu danh mục không tồn tại khi tạo món ăn mới. \\ \hline
    \textbf{Hậu điều kiện} & \multicolumn{3}{p{11.5cm}|}{Thực đơn được cập nhật thành công. Thay đổi được phản ánh ngay lập tức trên giao diện khách hàng thông qua FullMenuView. Sự kiện domain được ghi lại phục vụ phân tích.} \\ \hline
\end{longtable}

Dữ liệu đầu vào cho use case Quản lý thực đơn (Tạo món ăn mới):

\begin{longtable}{|c|p{3cm}|p{4cm}|c|p{3cm}|}
    \hline
    \textbf{STT} & \textbf{Trường dữ liệu} & \textbf{Mô tả} & \textbf{Bắt buộc?} & \textbf{Điều kiện hợp lệ} \\ \hline
    \endfirsthead
    
    \multicolumn{5}{c}{\textit{(Tiếp theo từ trang trước)}} \\
    \hline
    \textbf{STT} & \textbf{Trường dữ liệu} & \textbf{Mô tả} & \textbf{Bắt buộc?} & \textbf{Điều kiện hợp lệ} \\ \hline
    \endhead
    
    \hline
    \multicolumn{5}{r}{\textit{(Tiếp tục trang sau)}} \\
    \endfoot
    
    \caption{Dữ liệu đầu vào Quản lý thực đơn - Tạo món ăn} \label{tab:input_uc_008} \\
    \endlastfoot

    1 & RestaurantId & ID nhà hàng & Có & GUID hợp lệ \\ \hline
    2 & MenuCategoryId & ID danh mục món ăn & Có & GUID hợp lệ, tồn tại \\ \hline
    3 & Name & Tên món ăn & Có & Không rỗng, không chỉ khoảng trắng \\ \hline
    4 & Description & Mô tả chi tiết món & Có & Không rỗng, không chỉ khoảng trắng \\ \hline
    5 & Price & Giá cơ bản & Có & Số thực > 0 \\ \hline
    6 & Currency & Đơn vị tiền tệ & Có & Mã tiền tệ hợp lệ (VND, USD...) \\ \hline
    7 & ImageUrl & Đường dẫn ảnh món & Không & URL hợp lệ hoặc null \\ \hline
    8 & IsAvailable & Trạng thái còn hàng & Không & Boolean (mặc định: true) \\ \hline
    9 & DietaryTagIds & Danh sách tag dinh dưỡng & Không & Danh sách GUID (Vegetarian, Spicy...) \\ \hline
\end{longtable}


\clearpage
\subsection{Đặc tả Use Case: Tìm kiếm nhà hàng}
\begin{longtable}{|p{2.5cm}|p{1.5cm}|p{3cm}|p{5.5cm}|}
    \hline
    \textbf{Mã use case} & UC-002 & \textbf{Tên use case} & Tìm kiếm nhà hàng \\ \hline
    \textbf{Tác nhân} & \multicolumn{3}{p{11.5cm}|}{Khách hàng} \\ \hline
    \textbf{Mục đích sử dụng} & \multicolumn{3}{p{11.5cm}|}{Cho phép khách hàng tìm kiếm nhà hàng theo tên, món ăn, loại hình ẩm thực, vị trí địa lý và các bộ lọc khác (đánh giá, tag, khuyến mãi).} \\ \hline
    \textbf{Sự kiện kích hoạt} & \multicolumn{3}{p{11.5cm}|}{Khách hàng nhập từ khóa tìm kiếm hoặc chọn bộ lọc trên giao diện.} \\ \hline
    \textbf{Tiền điều kiện} & \multicolumn{3}{p{11.5cm}|}{Không yêu cầu đăng nhập. Hệ thống có dữ liệu nhà hàng đã được duyệt (IsVerified = true).} \\ \hline
    \endfirsthead
    
    \multicolumn{4}{r}{\textit{(Tiếp theo từ trang trước)}} \\
    \hline
    \endhead
    
    \hline
    \multicolumn{4}{r}{\textit{(Tiếp tục trang sau)}} \\
    \endfoot
    
    \caption{Đặc tả Use Case Tìm kiếm nhà hàng} \label{tab:uc_002} \\
    \endlastfoot

    \multirow{8}{2.5cm}{\textbf{Luồng sự kiện chính}} & \textbf{STT} & \textbf{Thực hiện bởi} & \textbf{Hành động} \\ \cline{2-4} 
     & 1 & Khách hàng & Nhập từ khóa tìm kiếm (tên nhà hàng/món ăn) hoặc chọn bộ lọc. \\ \cline{2-4} 
     & 2 & Hệ thống & Validate dữ liệu đầu vào (độ dài, định dạng tọa độ, phạm vi giá trị). \\ \cline{2-4} 
     & 3 & Hệ thống & Xây dựng câu truy vấn SQL với điều kiện lọc (WHERE clauses). \\ \cline{2-4} 
     & 4 & Hệ thống & Tính toán khoảng cách (nếu có tọa độ) bằng công thức Haversine. \\ \cline{2-4} 
     & 5 & Hệ thống & Sắp xếp kết quả theo tiêu chí (rating/distance/popularity/name). \\ \cline{2-4} 
     & 6 & Hệ thống & Phân trang kết quả (PageNumber, PageSize). \\ \cline{2-4} 
     & 7 & Hệ thống & Trả về danh sách nhà hàng và facets (nếu yêu cầu). \\ \hline
    \multirow{3}{2.5cm}{\textbf{Luồng sự kiện thay thế}} & \textbf{STT} & \textbf{Thực hiện bởi} & \textbf{Hành động} \\ \cline{2-4} 
     & 2a & Hệ thống & Trả về lỗi validation nếu tham số không hợp lệ (PageSize > 50, tọa độ ngoài phạm vi...). \\ \cline{2-4} 
     & 7a & Hệ thống & Trả về danh sách rỗng nếu không tìm thấy kết quả phù hợp. \\ \hline
    \textbf{Hậu điều kiện} & \multicolumn{3}{p{11.5cm}|}{Danh sách nhà hàng được hiển thị với thông tin: tên, logo, loại ẩm thực, đánh giá, khoảng cách (nếu có). Facets (bộ lọc động) được cập nhật dựa trên kết quả hiện tại.} \\ \hline
\end{longtable}

Dữ liệu đầu vào cho use case Tìm kiếm nhà hàng:

\begin{longtable}{|c|p{2.5cm}|p{3.5cm}|c|p{3.5cm}|}
    \hline
    \textbf{STT} & \textbf{Trường dữ liệu} & \textbf{Mô tả} & \textbf{Bắt buộc?} & \textbf{Điều kiện hợp lệ} \\ \hline
    \endfirsthead
    
    \multicolumn{5}{c}{\textit{(Tiếp theo từ trang trước)}} \\
    \hline
    \textbf{STT} & \textbf{Trường dữ liệu} & \textbf{Mô tả} & \textbf{Bắt buộc?} & \textbf{Điều kiện hợp lệ} \\ \hline
    \endhead
    
    \hline
    \multicolumn{5}{r}{\textit{(Tiếp tục trang sau)}} \\
    \endfoot
    
    \caption{Dữ liệu đầu vào Tìm kiếm nhà hàng} \label{tab:input_uc_002} \\
    \endlastfoot

    1 & Q & Từ khóa tìm kiếm & Không & Tối đa 100 ký tự \\ \hline
    2 & Cuisine & Loại ẩm thực & Không & Tối đa 50 ký tự \\ \hline
    3 & Lat & Vĩ độ & Không & -90 đến 90 \\ \hline
    4 & Lng & Kinh độ & Không & -180 đến 180 \\ \hline
    5 & MinRating & Đánh giá tối thiểu & Không & 0 đến 5 \\ \hline
    6 & Sort & Tiêu chí sắp xếp & Không & rating hoặc distance hoặc popularity \\ \hline
    7 & Bbox & Bounding box (bản đồ) & Không & minLon, minLat, maxLon, maxLat \\ \hline
    8 & Tags & Danh sách tag (tên) & Không & Tối đa 20 tag, mỗi tag \textless= 100 ký tự \\ \hline
    9 & TagIds & Danh sách tag (ID) & Không & Tối đa 20 GUID \\ \hline
    10 & DiscountedOnly & Chỉ nhà hàng có KM & Không & Boolean \\ \hline
    11 & PageNumber & Số trang & Có & >= 1 \\ \hline
    12 & PageSize & Kích thước trang & Có & 1 đến 50 \\ \hline
    13 & IncludeFacets & Bao gồm facets & Không & Boolean (mặc định: false) \\ \hline
\end{longtable}


\clearpage
\subsection{Đặc tả Use Case: Đặt hàng cá nhân}
\begin{longtable}{|p{2.5cm}|p{1.5cm}|p{3cm}|p{5.5cm}|}
    \hline
    \textbf{Mã use case} & UC-003 & \textbf{Tên use case} & Đặt hàng cá nhân \\ \hline
    \textbf{Tác nhân} & \multicolumn{3}{p{11.5cm}|}{Khách hàng} \\ \hline
    \textbf{Mục đích sử dụng} & \multicolumn{3}{p{11.5cm}|}{Cho phép khách hàng chọn món ăn, tùy chỉnh, áp dụng mã giảm giá và hoàn tất đơn hàng với thanh toán trực tuyến hoặc tiền mặt.} \\ \hline
    \textbf{Sự kiện kích hoạt} & \multicolumn{3}{p{11.5cm}|}{Khách hàng nhấn nút "Đặt hàng" sau khi đã chọn món và điền đầy đủ thông tin giao hàng.} \\ \hline
    \textbf{Tiền điều kiện} & \multicolumn{3}{p{11.5cm}|}{Khách hàng đã đăng nhập. Nhà hàng đang hoạt động. Giỏ hàng có ít nhất 1 món.} \\ \hline
    \endfirsthead
    
    \multicolumn{4}{r}{\textit{(Tiếp theo từ trang trước)}} \\
    \hline
    \endhead
    
    \hline
    \multicolumn{4}{r}{\textit{(Tiếp tục trang sau)}} \\
    \endfoot
    
    \caption{Đặc tả Use Case Đặt hàng cá nhân} \label{tab:uc_003} \\
    \endlastfoot
    
    \multirow{13}{2.5cm}{\textbf{Luồng sự kiện chính}} & \textbf{STT} & \textbf{Thực hiện bởi} & \textbf{Hành động} \\ \cline{2-4} 
     & 1 & Khách hàng & Chọn món ăn từ thực đơn, tùy chỉnh (size, topping) nếu có. \\ \cline{2-4} 
     & 2 & Khách hàng & Thêm món vào giỏ hàng (tối đa 50 món, mỗi món tối đa 10 phần). \\ \cline{2-4} 
     & 3 & Khách hàng & Chọn địa chỉ giao hàng từ danh sách hoặc nhập mới. \\ \cline{2-4} 
     & 4 & Khách hàng & Nhập mã coupon (tùy chọn) và số tiền tip (tùy chọn). \\ \cline{2-4} 
     & 5 & Hệ thống & Kiểm tra tính hợp lệ: nhà hàng hoạt động, món ăn còn hàng, thuộc đúng nhà hàng. \\ \cline{2-4} 
     & 6 & Hệ thống & Kiểm tra tùy chỉnh: nhóm tùy chỉnh được gán cho món, số lượng lựa chọn hợp lệ (min-max). \\ \cline{2-4} 
     & 7 & Hệ thống & Tính toán tài chính: Subtotal, Discount (nếu có coupon), DeliveryFee, Tax, Tip, TotalAmount. \\ \cline{2-4} 
     & 8 & Hệ thống & Kiểm tra và tăng số lần sử dụng coupon (nếu có) trong cùng transaction. \\ \cline{2-4} 
     & 9 & Khách hàng & Chọn phương thức thanh toán (CreditCard, PayPal, ApplePay, GooglePay, COD). \\ \cline{2-4} 
     & 10 & Hệ thống & Tạo PaymentIntent (Stripe) nếu thanh toán online, lưu PaymentIntentId vào Order. \\ \cline{2-4} 
     & 11 & Hệ thống & Tạo Order với trạng thái PendingPayment (online) hoặc Placed (COD). \\ \cline{2-4} 
     & 12 & Hệ thống & Lưu Order vào database, phát sự kiện OrderPlaced. \\ \cline{2-4} 
     & 13 & Hệ thống & Trả về OrderId, OrderNumber, TotalAmount và ClientSecret (nếu online). \\ \hline
    \multirow{6}{2.5cm}{\textbf{Luồng sự kiện thay thế}} & \textbf{STT} & \textbf{Thực hiện bởi} & \textbf{Hành động} \\ \cline{2-4} 
     & 5a & Hệ thống & Báo lỗi nếu nhà hàng không tồn tại hoặc không hoạt động. \\ \cline{2-4} 
     & 5b & Hệ thống & Báo lỗi nếu món ăn không tồn tại, không còn hàng hoặc không thuộc nhà hàng. \\ \cline{2-4} 
     & 6a & Hệ thống & Báo lỗi nếu nhóm tùy chỉnh không được gán cho món hoặc số lượng lựa chọn sai. \\ \cline{2-4} 
     & 8a & Hệ thống & Báo lỗi nếu coupon không hợp lệ, hết hạn hoặc đã hết lượt sử dụng. \\ \cline{2-4} 
     & 10a & Hệ thống & Báo lỗi nếu không tạo được PaymentIntent (lỗi Stripe API). \\ \hline
    \textbf{Hậu điều kiện} & \multicolumn{3}{p{11.5cm}|}{Đơn hàng được tạo thành công với trạng thái phù hợp. Sự kiện OrderPlaced được phát ra để thông báo cho nhà hàng. Nếu thanh toán online, khách hàng được chuyển đến trang thanh toán Stripe. Nếu COD, đơn hàng chuyển trạng thái Placed ngay lập tức.} \\ \hline
\end{longtable}

Dữ liệu đầu vào cho use case Đặt hàng cá nhân:

\begin{longtable}{|c|p{4.0cm}|p{3.0cm}|c|p{3.5cm}|}
    \hline
    \textbf{STT} & \textbf{Trường dữ liệu} & \textbf{Mô tả} & \textbf{Bắt buộc?} & \textbf{Điều kiện hợp lệ} \\ \hline
    \endfirsthead
    
    \multicolumn{5}{c}{\textit{(Tiếp theo từ trang trước)}} \\
    \hline
    \textbf{STT} & \textbf{Trường dữ liệu} & \textbf{Mô tả} & \textbf{Bắt buộc?} & \textbf{Điều kiện hợp lệ} \\ \hline
    \endhead
    
    \hline
    \multicolumn{5}{r}{\textit{(Tiếp tục trang sau)}} \\
    \endfoot
    
    \caption{Dữ liệu đầu vào Đặt hàng cá nhân} \label{tab:input_uc_003} \\
    \endlastfoot

    1 & CustomerId & ID khách hàng & Có & GUID hợp lệ, trùng với user hiện tại \\ \hline
    2 & RestaurantId & ID nhà hàng & Có & GUID hợp lệ, nhà hàng tồn tại \\ \hline
    3 & Items & Danh sách món ăn & Có & Tối thiểu 1, tối đa 50 món \\ \hline
    4 & Items[].\hspace{0pt}MenuItemId & ID món ăn & Có & GUID hợp lệ, món tồn tại \\ \hline
    5 & Items[].\hspace{0pt}Quantity & Số lượng & Có & 1 đến 10 \\ \hline
    6 & Items[].\hspace{0pt}Customizations & Danh sách tùy chỉnh & Không & Phụ thuộc món ăn \\ \hline
    7 & DeliveryAddress.\newline Street & Số nhà, đường & Có & Tối đa 200 ký tự \\ \hline
    8 & DeliveryAddress.\newline City & Thành phố & Có & Tối đa 100 ký tự \\ \hline
    9 & DeliveryAddress.\newline State & Tỉnh/Bang & Có & Tối đa 100 ký tự \\ \hline
    10 & DeliveryAddress.\newline ZipCode & Mã bưu điện & Có & Tối đa 20 ký tự \\ \hline
    11 & DeliveryAddress.\newline Country & Quốc gia & Có & Tối đa 100 ký tự \\ \hline
    12 & PaymentMethod & Phương thức thanh toán & Có & CreditCard, PayPal, ApplePay, GooglePay, COD \\ \hline
    13 & Special\newline Instructions & Ghi chú đặc biệt & Không & Tối đa 500 ký tự \\ \hline
    14 & CouponCode & Mã giảm giá & Không & Tối đa 50 ký tự \\ \hline
    15 & TipAmount & Tiền tip & Không & >= 0 \\ \hline
    16 & TeamCartId & ID giỏ hàng nhóm & Không & GUID hợp lệ (nếu từ TeamCart) \\ \hline
\end{longtable}


\clearpage
\subsection{Đặc tả Use Case: Khởi tạo TeamCart}
\begin{longtable}{|p{2.5cm}|p{1.5cm}|p{3cm}|p{5.5cm}|}
    \hline
    \textbf{Mã use case} & UC-004a & \textbf{Tên use case} & Khởi tạo TeamCart \\ \hline
    \textbf{Tác nhân} & \multicolumn{3}{p{11.5cm}|}{Khách hàng (Host)} \\ \hline
    \textbf{Mục đích sử dụng} & \multicolumn{3}{p{11.5cm}|}{Cho phép khách hàng tạo một giỏ hàng nhóm (TeamCart) để mời những người khác cùng đặt món từ một nhà hàng cụ thể.} \\ \hline
    \textbf{Sự kiện kích hoạt} & \multicolumn{3}{p{11.5cm}|}{Khách hàng chọn chức năng "Đặt nhóm" (Group Order) trên giao diện chi tiết nhà hàng.} \\ \hline
    \textbf{Tiền điều kiện} & \multicolumn{3}{p{11.5cm}|}{Khách hàng đã đăng nhập. Nhà hàng tồn tại và đang hoạt động.} \\ \hline
    \endfirsthead
    
    \multicolumn{4}{r}{\textit{(Tiếp theo từ trang trước)}} \\
    \hline
    \endhead
    
    \hline
    \multicolumn{4}{r}{\textit{(Tiếp tục trang sau)}} \\
    \endfoot
    
    \caption{Đặc tả Use Case Khởi tạo TeamCart} \label{tab:uc_004a} \\
    \endlastfoot

    \multirow{5}{2.5cm}{\textbf{Luồng sự kiện chính}} & \textbf{STT} & \textbf{Thực hiện bởi} & \textbf{Hành động} \\ \cline{2-4} 
     & 1 & Khách hàng & Chọn nhà hàng và thiết lập thông tin giỏ nhóm (Tên hiển thị của Host, Thời gian chốt đơn). \\ \cline{2-4} 
     & 2 & Hệ thống & Kiểm tra tính hợp lệ của dữ liệu (Tên host không rỗng, thời gian chốt đơn phải ở tương lai nếu có). \\ \cline{2-4} 
     & 3 & Hệ thống & Tạo mới TeamCart với trạng thái Active. Tạo ShareToken duy nhất. \\ \cline{2-4} 
     & 4 & Hệ thống & Thêm người tạo vào danh sách thành viên với vai trò Host. \\ \cline{2-4} 
     & 5 & Hệ thống & Trả về thông tin TeamCart và Link chia sẻ (ShareToken). \\ \hline
    \multirow{2}{2.5cm}{\textbf{Luồng sự kiện thay thế}} & \textbf{STT} & \textbf{Thực hiện bởi} & \textbf{Hành động} \\ \cline{2-4} 
     & 2a & Hệ thống & Báo lỗi nếu nhà hàng không tồn tại hoặc thời gian chốt đơn không hợp lệ (trong quá khứ). \\ \hline
    \textbf{Hậu điều kiện} & \multicolumn{3}{p{11.5cm}|}{TeamCart được tạo thành công. Host nhận được link chia sẻ để gửi cho các thành viên khác.} \\ \hline
\end{longtable}

Dữ liệu đầu vào cho use case Khởi tạo TeamCart:

\begin{longtable}{|c|p{3cm}|p{4cm}|c|p{3cm}|}
    \hline
    \textbf{STT} & \textbf{Trường dữ liệu} & \textbf{Mô tả} & \textbf{Bắt buộc?} & \textbf{Điều kiện hợp lệ} \\ \hline
    \endfirsthead
    
    \multicolumn{5}{c}{\textit{(Tiếp theo từ trang trước)}} \\
    \hline
    \textbf{STT} & \textbf{Trường dữ liệu} & \textbf{Mô tả} & \textbf{Bắt buộc?} & \textbf{Điều kiện hợp lệ} \\ \hline
    \endhead
    
    \hline
    \multicolumn{5}{r}{\textit{(Tiếp tục trang sau)}} \\
    \endfoot
    
    \caption{Dữ liệu đầu vào Khởi tạo TeamCart} \label{tab:input_uc_004a} \\
    \endlastfoot

    1 & RestaurantId & ID nhà hàng & Có & GUID hợp lệ, nhà hàng tồn tại \\ \hline
    2 & HostName & Tên hiển thị của Host & Có & Không rỗng, tối đa 200 ký tự \\ \hline
    3 & DeadlineUtc & Thời gian chốt đơn & Không & Thời gian trong tương lai (nếu có) \\ \hline
\end{longtable}


\clearpage
\subsection{Đặc tả Use Case: Tham gia TeamCart}
\begin{longtable}{|p{2.5cm}|p{1.5cm}|p{3cm}|p{5.5cm}|}
    \hline
    \textbf{Mã use case} & UC-004b & \textbf{Tên use case} & Tham gia TeamCart \\ \hline
    \textbf{Tác nhân} & \multicolumn{3}{p{11.5cm}|}{Khách hàng (Member)} \\ \hline
    \textbf{Mục đích sử dụng} & \multicolumn{3}{p{11.5cm}|}{Cho phép khách hàng tham gia vào một giỏ hàng nhóm thông qua liên kết chia sẻ để cùng đặt món.} \\ \hline
    \textbf{Sự kiện kích hoạt} & \multicolumn{3}{p{11.5cm}|}{Khách hàng truy cập vào đường dẫn chia sẻ (Share Link) của TeamCart.} \\ \hline
    \textbf{Tiền điều kiện} & \multicolumn{3}{p{11.5cm}|}{Khách hàng đã đăng nhập. TeamCart tồn tại, đang hoạt động (Active) và chưa bị khóa. Mã chia sẻ (ShareToken) hợp lệ.} \\ \hline
    \endfirsthead
    
    \multicolumn{4}{r}{\textit{(Tiếp theo từ trang trước)}} \\
    \hline
    \endhead
    
    \hline
    \multicolumn{4}{r}{\textit{(Tiếp tục trang sau)}} \\
    \endfoot
    
    \caption{Đặc tả Use Case Tham gia TeamCart} \label{tab:uc_004b} \\
    \endlastfoot

    \multirow{6}{2.5cm}{\textbf{Luồng sự kiện chính}} & \textbf{STT} & \textbf{Thực hiện bởi} & \textbf{Hành động} \\ \cline{2-4} 
     & 1 & Khách hàng & Nhập tên hiển thị (GuestName) để tham gia nhóm. \\ \cline{2-4} 
     & 2 & Hệ thống & Kiểm tra sự tồn tại của TeamCart. \\ \cline{2-4} 
     & 3 & Hệ thống & Xác thực mã chia sẻ (ShareToken). \\ \cline{2-4} 
     & 4 & Hệ thống & Kiểm tra trạng thái TeamCart (phải là Active) và quyền tham gia (chưa tham gia trước đó). \\ \cline{2-4} 
     & 5 & Hệ thống & Thêm người dùng vào danh sách thành viên với vai trò Guest. \\ \cline{2-4} 
     & 6 & Hệ thống & Thông báo thành công và chuyển hướng đến giao diện chi tiết TeamCart. \\ \hline
    \multirow{3}{2.5cm}{\textbf{Luồng sự kiện thay thế}} & \textbf{STT} & \textbf{Thực hiện bởi} & \textbf{Hành động} \\ \cline{2-4} 
     & 2a & Hệ thống & Báo lỗi nếu TeamCart không tồn tại. \\ \cline{2-4} 
     & 3a & Hệ thống & Báo lỗi nếu mã chia sẻ không hợp lệ hoặc đã hết hạn. \\ \cline{2-4} 
     & 4a & Hệ thống & Báo lỗi nếu TeamCart đã bị khóa, đã chốt đơn hoặc người dùng đã là thành viên. \\ \hline
    \textbf{Hậu điều kiện} & \multicolumn{3}{p{11.5cm}|}{Người dùng trở thành thành viên của TeamCart, có quyền thêm món ăn vào giỏ chung.} \\ \hline
\end{longtable}

Dữ liệu đầu vào cho use case Tham gia TeamCart:

\begin{longtable}{|c|p{3cm}|p{4cm}|c|p{3cm}|}
    \hline
    \textbf{STT} & \textbf{Trường dữ liệu} & \textbf{Mô tả} & \textbf{Bắt buộc?} & \textbf{Điều kiện hợp lệ} \\ \hline
    \endfirsthead
    
    \multicolumn{5}{c}{\textit{(Tiếp theo từ trang trước)}} \\
    \hline
    \textbf{STT} & \textbf{Trường dữ liệu} & \textbf{Mô tả} & \textbf{Bắt buộc?} & \textbf{Điều kiện hợp lệ} \\ \hline
    \endhead
    
    \hline
    \multicolumn{5}{r}{\textit{(Tiếp tục trang sau)}} \\
    \endfoot
    
    \caption{Dữ liệu đầu vào Tham gia TeamCart} \label{tab:input_uc_004b} \\
    \endlastfoot

    1 & TeamCartId & ID giỏ hàng nhóm & Có & GUID hợp lệ, tồn tại \\ \hline
    2 & ShareToken & Mã chia sẻ & Có & Tối đa 100 ký tự, khớp với TeamCart \\ \hline
    3 & GuestName & Tên hiển thị & Có & Không rỗng, tối đa 200 ký tự \\ \hline
\end{longtable}


\clearpage
\subsection{Đặc tả Use Case: Chốt đơn TeamCart}
\begin{longtable}{|p{2.5cm}|p{1.5cm}|p{3cm}|p{5.5cm}|}
    \hline
    \textbf{Mã use case} & UC-004c & \textbf{Tên use case} & Chốt đơn TeamCart \\ \hline
    \textbf{Tác nhân} & \multicolumn{3}{p{11.5cm}|}{Khách hàng (Host)} \\ \hline
    \textbf{Mục đích sử dụng} & \multicolumn{3}{p{11.5cm}|}{Cho phép Host khóa giỏ hàng, chờ các thành viên thanh toán và hoàn tất việc chuyển đổi TeamCart thành đơn hàng chính thức.} \\ \hline
    \textbf{Sự kiện kích hoạt} & \multicolumn{3}{p{11.5cm}|}{Host nhấn nút "Đặt hàng" sau khi đã đủ điều kiện.} \\ \hline
    \textbf{Tiền điều kiện} & \multicolumn{3}{p{11.5cm}|}{TeamCart đang hoạt động (Active) và có món ăn. Host đã đăng nhập.} \\ \hline
    \endfirsthead
    
    \multicolumn{4}{r}{\textit{(Tiếp theo từ trang trước)}} \\
    \hline
    \endhead
    
    \hline
    \multicolumn{4}{r}{\textit{(Tiếp tục trang sau)}} \\
    \endfoot
    
    \caption{Đặc tả Use Case Chốt đơn TeamCart} \label{tab:uc_004c} \\
    \endlastfoot

    \multirow{9}{2.5cm}{\textbf{Luồng sự kiện chính}} & \textbf{STT} & \textbf{Thực hiện bởi} & \textbf{Hành động} \\ \cline{2-4} 
     & 1 & Host & Chọn chức năng "Khóa đơn" để ngăn chặn việc thêm/bớt món. \\ \cline{2-4} 
     & 2 & Hệ thống & Tính toán chi phí (Tổng tiền, Thuế, Phí vận chuyển) và chia tiền cho từng thành viên. Chuyển trạng thái sang "Locked". \\ \cline{2-4} 
     & 3 & Thành viên & Thực hiện thanh toán phần tiền của mình (thanh toán online hoặc COD). \\ \cline{2-4} 
     & 4 & Hệ thống & Cập nhật trạng thái thanh toán. Khi đủ tiền, chuyển trạng thái TeamCart sang "ReadyToConfirm". \\ \cline{2-4} 
     & 5 & Host & Nhập thông tin giao hàng và xác nhận đặt hàng cuối cùng. \\ \cline{2-4} 
     & 6 & Hệ thống & Kiểm tra tính nhất quán của dữ liệu (Quote Version, Tổng tiền đã đóng). \\ \cline{2-4} 
     & 7 & Hệ thống & Kiểm tra và chốt hạn mức sử dụng Coupon (nếu có). \\ \cline{2-4} 
     & 8 & Hệ thống & Tạo đơn hàng (Order) mới và chuyển trạng thái TeamCart sang "Converted". \\ \cline{2-4} 
     & 9 & Hệ thống & Thông báo đặt hàng thành công cho tất cả thành viên. \\ \hline
    \multirow{3}{2.5cm}{\textbf{Luồng sự kiện thay thế}} & \textbf{STT} & \textbf{Thực hiện bởi} & \textbf{Hành động} \\ \cline{2-4} 
     & 6a & Hệ thống & Báo lỗi nếu thông tin báo giá (Quote) đã thay đổi hoặc không khớp. \\ \cline{2-4} 
     & 6b & Hệ thống & Báo lỗi nếu tổng số tiền thanh toán từ các thành viên chưa đủ khớp với tổng đơn hàng. \\ \cline{2-4} 
     & 7a & Hệ thống & Báo lỗi nếu Coupon hết lượt sử dụng tại thời điểm chốt đơn. \\ \hline
    \textbf{Hậu điều kiện} & \multicolumn{3}{p{11.5cm}|}{Đơn hàng được tạo thành công trên hệ thống. TeamCart hoàn tất vòng đời.} \\ \hline
\end{longtable}

Dữ liệu đầu vào cho use case Chốt đơn TeamCart (Bước xác nhận cuối cùng):

\begin{longtable}{|c|p{3cm}|p{4cm}|c|p{3cm}|}
    \hline
    \textbf{STT} & \textbf{Trường dữ liệu} & \textbf{Mô tả} & \textbf{Bắt buộc?} & \textbf{Điều kiện hợp lệ} \\ \hline
    \endfirsthead
    
    \multicolumn{5}{c}{\textit{(Tiếp theo từ trang trước)}} \\
    \hline
    \textbf{STT} & \textbf{Trường dữ liệu} & \textbf{Mô tả} & \textbf{Bắt buộc?} & \textbf{Điều kiện hợp lệ} \\ \hline
    \endhead
    
    \hline
    \multicolumn{5}{r}{\textit{(Tiếp tục trang sau)}} \\
    \endfoot
    
    \caption{Dữ liệu đầu vào Chốt đơn TeamCart} \label{tab:input_uc_004c} \\
    \endlastfoot

    1 & TeamCartId & ID giỏ hàng nhóm & Có & GUID hợp lệ, trạng thái ReadyToConfirm \\ \hline
    2 & Street & Số nhà, tên đường & Có & Tối đa 200 ký tự \\ \hline
    3 & City & Thành phố & Có & Tối đa 100 ký tự \\ \hline
    4 & State & Tỉnh/Bang & Có & Tối đa 100 ký tự \\ \hline
    5 & ZipCode & Mã bưu điện & Có & Tối đa 20 ký tự \\ \hline
    6 & Country & Quốc gia & Có & Tối đa 100 ký tự \\ \hline
    7 & Special\newline Instructions & Ghi chú đơn hàng & Không & Tối đa 500 ký tự \\ \hline
    8 & QuoteVersion & Phiên bản báo giá & Có & Khớp với phiên bản hiện tại của Cart \\ \hline
\end{longtable}


\clearpage
\subsection{Đặc tả Use Case: Xử lý đơn hàng}
\begin{longtable}{|p{2.5cm}|p{1.5cm}|p{3cm}|p{5.5cm}|}
    \hline
    \textbf{Mã use case} & UC-009 & \textbf{Tên use case} & Xử lý đơn hàng \\ \hline
    \textbf{Tác nhân} & \multicolumn{3}{p{11.5cm}|}{Nhà hàng (Restaurant Staff)} \\ \hline
    \textbf{Mục đích sử dụng} & \multicolumn{3}{p{11.5cm}|}{Cho phép nhà hàng tiếp nhận, xử lý và cập nhật trạng thái đơn hàng theo quy trình từ lúc nhận đơn đến khi giao thành công.} \\ \hline
    \textbf{Sự kiện kích hoạt} & \multicolumn{3}{p{11.5cm}|}{Nhà hàng nhận được thông báo có đơn hàng mới (trạng thái Placed) trên giao diện quản lý.} \\ \hline
    \textbf{Tiền điều kiện} & \multicolumn{3}{p{11.5cm}|}{Đơn hàng tồn tại và đang ở trạng thái chờ xử lý (Placed). Người dùng có quyền quản lý đơn hàng của nhà hàng.} \\ \hline
    \endfirsthead
    
    \multicolumn{4}{r}{\textit{(Tiếp theo từ trang trước)}} \\
    \hline
    \endhead
    
    \hline
    \multicolumn{4}{r}{\textit{(Tiếp tục trang sau)}} \\
    \endfoot
    
    \caption{Đặc tả Use Case Xử lý đơn hàng} \label{tab:uc_009} \\
    \endlastfoot

    \multirow{9}{2.5cm}{\textbf{Luồng sự kiện chính}} & \textbf{STT} & \textbf{Thực hiện bởi} & \textbf{Hành động} \\ \cline{2-4} 
     & 1 & Nhà hàng & Xem chi tiết đơn hàng mới và chọn "Chấp nhận" (Accept). \\ \cline{2-4} 
     & 2 & Nhà hàng & Nhập thời gian giao hàng dự kiến (Estimated Delivery Time). \\ \cline{2-4} 
     & 3 & Hệ thống & Kiểm tra trạng thái đơn hàng, cập nhật sang "Accepted" và thông báo cho khách hàng. \\ \cline{2-4} 
     & 4 & Nhà hàng & Chuyển trạng thái sang "Đang chuẩn bị" (Preparing) khi bắt đầu chế biến. \\ \cline{2-4} 
     & 5 & Hệ thống & Cập nhật trạng thái sang "Preparing" và thông báo tiến độ. \\ \cline{2-4} 
     & 6 & Nhà hàng & Chuyển trạng thái sang "Sẵn sàng giao" (ReadyForDelivery) khi món ăn hoàn tất. \\ \cline{2-4} 
     & 7 & Hệ thống & Cập nhật trạng thái sang "ReadyForDelivery", thông báo cho tài xế hoặc khách hàng đến lấy. \\ \cline{2-4} 
     & 8 & Nhà hàng & Xác nhận "Đã giao" (Delivered) khi đơn hàng được giao thành công cho khách. \\ \cline{2-4} 
     & 9 & Hệ thống & Cập nhật trạng thái sang "Delivered", ghi nhận doanh thu và hoàn tất đơn hàng. \\ \hline
    \multirow{3}{2.5cm}{\textbf{Luồng sự kiện thay thế}} & \textbf{STT} & \textbf{Thực hiện bởi} & \textbf{Hành động} \\ \cline{2-4} 
     & 1a & Nhà hàng & Chọn "Từ chối" (Reject) đơn hàng. \\ \cline{2-4} 
     & 1b & Nhà hàng & Nhập lý do từ chối (hết món, quá tải...). \\ \cline{2-4} 
     & 3a & Hệ thống & Cập nhật trạng thái sang "Rejected", hoàn tiền cho khách (nếu đã thanh toán) và gửi thông báo hủy. \\ \hline
    \textbf{Hậu điều kiện} & \multicolumn{3}{p{11.5cm}|}{Đơn hàng kết thúc ở trạng thái Delivered (thành công) hoặc Rejected (hủy bỏ). Lịch sử trạng thái được ghi lại đầy đủ.} \\ \hline
\end{longtable}

Dữ liệu đầu vào cho use case Xử lý đơn hàng (Các thao tác chính):

\begin{longtable}{|c|p{3cm}|p{4cm}|c|p{3cm}|}
    \hline
    \textbf{STT} & \textbf{Trường dữ liệu} & \textbf{Mô tả} & \textbf{Bắt buộc?} & \textbf{Điều kiện hợp lệ} \\ \hline
    \endfirsthead
    
    \multicolumn{5}{c}{\textit{(Tiếp theo từ trang trước)}} \\
    \hline
    \textbf{STT} & \textbf{Trường dữ liệu} & \textbf{Mô tả} & \textbf{Bắt buộc?} & \textbf{Điều kiện hợp lệ} \\ \hline
    \endhead
    
    \hline
    \multicolumn{5}{r}{\textit{(Tiếp tục trang sau)}} \\
    \endfoot
    
    \caption{Dữ liệu đầu vào Xử lý đơn hàng} \label{tab:input_uc_009} \\
    \endlastfoot

    1 & OrderId & ID đơn hàng & Có & GUID hợp lệ, tồn tại \\ \hline
    2 & Action & Hành động xử lý & Có & Accept, Reject, MarkPreparing, MarkReady, MarkDelivered \\ \hline
    3 & Estimated\newline DeliveryTime & Thời gian giao dự kiến & Có (khi Accept) & Thời gian > Hiện tại \\ \hline
    4 & RejectionReason & Lý do từ chối & Có (khi Reject) & Tối đa 200 ký tự \\ \hline
\end{longtable}


\clearpage
\subsection{Đặc tả Use Case: Đánh giá nhà hàng}
\begin{longtable}{|p{2.5cm}|p{1.5cm}|p{3cm}|p{5.5cm}|}
    \hline
    \textbf{Mã use case} & UC-006 & \textbf{Tên use case} & Đánh giá nhà hàng \\ \hline
    \textbf{Tác nhân} & \multicolumn{3}{p{11.5cm}|}{Khách hàng} \\ \hline
    \textbf{Mục đích sử dụng} & \multicolumn{3}{p{11.5cm}|}{Cho phép khách hàng đánh giá chất lượng dịch vụ và món ăn của nhà hàng sau khi hoàn tất đơn hàng.} \\ \hline
    \textbf{Sự kiện kích hoạt} & \multicolumn{3}{p{11.5cm}|}{Khách hàng chọn chức năng "Đánh giá" trên lịch sử đơn hàng đã giao thành công.} \\ \hline
    \textbf{Tiền điều kiện} & \multicolumn{3}{p{11.5cm}|}{Khách hàng đã đăng nhập. Đơn hàng đã hoàn tất (Delivered). Khách hàng chưa từng đánh giá nhà hàng này trước đó (hoặc hệ thống cho phép đánh giá lại).} \\ \hline
    \endfirsthead
    
    \multicolumn{4}{r}{\textit{(Tiếp theo từ trang trước)}} \\
    \hline
    \endhead
    
    \hline
    \multicolumn{4}{r}{\textit{(Tiếp tục trang sau)}} \\
    \endfoot
    
    \caption{Đặc tả Use Case Đánh giá nhà hàng} \label{tab:uc_006} \\
    \endlastfoot

    \multirow{5}{2.5cm}{\textbf{Luồng sự kiện chính}} & \textbf{STT} & \textbf{Thực hiện bởi} & \textbf{Hành động} \\ \cline{2-4} 
     & 1 & Khách hàng & Chọn mức đánh giá (số sao từ 1 đến 5) và nhập nội dung bình luận (tùy chọn). \\ \cline{2-4} 
     & 2 & Hệ thống & Kiểm tra tính hợp lệ của dữ liệu (số sao trong phạm vi, độ dài bình luận). \\ \cline{2-4} 
     & 3 & Hệ thống & Kiểm tra điều kiện nghiệp vụ: Đơn hàng phải thuộc về khách hàng và đã giao thành công. \\ \cline{2-4} 
     & 4 & Hệ thống & Kiểm tra trùng lặp: Đảm bảo khách hàng chưa đánh giá đơn hàng này (hoặc nhà hàng này) trước đó. \\ \cline{2-4} 
     & 5 & Hệ thống & Lưu đánh giá vào hệ thống, cập nhật điểm đánh giá trung bình của nhà hàng và hiển thị thông báo thành công. \\ \hline
    \multirow{3}{2.5cm}{\textbf{Luồng sự kiện thay thế}} & \textbf{STT} & \textbf{Thực hiện bởi} & \textbf{Hành động} \\ \cline{2-4} 
     & 2a & Hệ thống & Báo lỗi nếu số sao không hợp lệ hoặc nội dung quá dài. \\ \cline{2-4} 
     & 3a & Hệ thống & Báo lỗi nếu đơn hàng chưa hoàn tất hoặc không tồn tại. \\ \cline{2-4} 
     & 4a & Hệ thống & Báo lỗi nếu khách hàng đã đánh giá trước đó (ReviewAlreadyExists). \\ \hline
    \textbf{Hậu điều kiện} & \multicolumn{3}{p{11.5cm}|}{Đánh giá được hiển thị công khai trên trang chi tiết nhà hàng. Điểm số trung bình của nhà hàng được tính toán lại.} \\ \hline
\end{longtable}

Dữ liệu đầu vào cho use case Đánh giá nhà hàng:

\begin{longtable}{|c|p{3cm}|p{4cm}|c|p{3cm}|}
    \hline
    \textbf{STT} & \textbf{Trường dữ liệu} & \textbf{Mô tả} & \textbf{Bắt buộc?} & \textbf{Điều kiện hợp lệ} \\ \hline
    \endfirsthead
    
    \multicolumn{5}{c}{\textit{(Tiếp theo từ trang trước)}} \\
    \hline
    \textbf{STT} & \textbf{Trường dữ liệu} & \textbf{Mô tả} & \textbf{Bắt buộc?} & \textbf{Điều kiện hợp lệ} \\ \hline
    \endhead
    
    \hline
    \multicolumn{5}{r}{\textit{(Tiếp tục trang sau)}} \\
    \endfoot
    
    \caption{Dữ liệu đầu vào Đánh giá nhà hàng} \label{tab:input_uc_006} \\
    \endlastfoot

    1 & OrderId & ID đơn hàng & Có & GUID hợp lệ, đã giao \\ \hline
    2 & RestaurantId & ID nhà hàng & Có & GUID hợp lệ \\ \hline
    3 & Rating & Điểm đánh giá & Có & Số nguyên từ 1 đến 5 \\ \hline
    4 & Title & Tiêu đề đánh giá & Không & Tối đa 100 ký tự \\ \hline
    5 & Comment & Nội dung chi tiết & Không & Tối đa 1000 ký tự \\ \hline
\end{longtable}

\end{document}
