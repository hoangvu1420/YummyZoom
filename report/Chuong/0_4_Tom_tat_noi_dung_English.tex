\documentclass[../DoAn.tex]{subfiles}
\begin{document}

\begin{center}
    \Large{\textbf{ABSTRACT}}\\
\end{center}
\vspace{1cm}
Group food ordering has become a routine practice among students and office workers, particularly during limited time windows such as lunch breaks. However, despite the convenience of modern food delivery platforms, group ordering remains hampered by a significant pain point: the payment process typically requires one person to pay for the entire order upfront and then collect money from each member afterward. This approach is time-consuming, prone to errors, and creates unnecessary friction outside the app. To address this problem, this thesis presents \gls{yummyzoom}, a food delivery system developed as a Minimum Viable Product (\gls{mvp}) with a key focus on the \gls{teamcart} feature. \gls{teamcart} allows multiple users to collaboratively add items to a shared cart, receive real-time updates, and most importantly, enables each member to pay for their own portion independently.

From a technical perspective, the project leverages \gls{cleanarchitecture} combined with Domain-Driven Design (\gls{ddd}) to model the business domain and manage the complex collaboration rules within \gls{teamcart}. The application layer follows a \gls{cqrs} pattern to clearly separate read and write operations. The backend is built on \gls{dotnet} 9 and \gls{aspnetcore}, with \gls{postgresql} serving as the primary data store, \gls{redis} providing fast caching, and \gls{signalr} handling real-time synchronization. The mobile application is developed using \gls{flutter} to ensure a consistent user experience across both iOS and Android platforms. Additionally, the system includes web portals for restaurants and administrators, built with \gls{angular} and PrimeNG, to manage menus and monitor orders.

The main contributions of this work include a well-structured, testable software architecture and an effective solution for handling concurrent updates in \gls{teamcart} through \gls{optimisticconcurrency} and real-time synchronization, minimizing conflicts when multiple users interact simultaneously. The resulting system fully supports both individual and group ordering workflows, validated through scenario-based testing and experimental deployment. The thesis also acknowledges several limitations that require future work, including the lack of production-ready payment gateway integration, the absence of a driver management module, missing AI-powered search and recommendation features, and the need for a comprehensive financial management system for restaurants.

\end{document}
