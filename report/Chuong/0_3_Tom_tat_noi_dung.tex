\documentclass[../DoAn.tex]{subfiles}
\begin{document}

\begin{center}
    \Large{\textbf{TÓM TẮT NỘI DUNG ĐỒ ÁN}}\\
\end{center}
\vspace{1cm}
Việc đặt đồ ăn chung trong nhóm bạn bè hay đồng nghiệp là chuyện thường ngày, nhất là vào giờ trưa khi thời gian có hạn. Tuy nhiên, dù các ứng dụng giao đồ ăn hiện nay đã khá tiện lợi, việc đặt hàng nhóm vẫn còn một vấn đề nhức nhối: thường phải có một người trả tiền trước rồi sau đó đi thu từng người một. Chuyện này không chỉ mất thời gian mà còn phiền toái, dễ quên hoặc tính sai tiền. Xuất phát từ thực tế đó, đồ án này xây dựng YummyZoom – một hệ thống giao đồ ăn ở mức MVP (Minimum Viable Product) với điểm nhấn là tính năng TeamCart (Giỏ hàng nhóm). TeamCart cho phép nhiều người cùng lúc chọn món vào chung một giỏ hàng, theo dõi những thay đổi ngay lập tức, và quan trọng nhất là mỗi người tự thanh toán phần của mình – không cần ai đứng ra "bao" cả nhóm nữa.

Về mặt kỹ thuật, đồ án áp dụng Clean Architecture kết hợp với Domain-Driven Design (DDD) để xây dựng mô hình nghiệp vụ và quản lý các quy tắc phức tạp trong TeamCart. Lớp ứng dụng được tổ chức theo mô hình CQRS, giúp tách biệt rõ ràng giữa việc đọc và ghi dữ liệu. Phía backend sử dụng .NET 9 và ASP.NET Core, dữ liệu chính lưu trên PostgreSQL, còn Redis đảm nhận việc cache nhanh và SignalR lo phần đồng bộ real-time. Ứng dụng di động được viết bằng Flutter để chạy mượt mà trên cả iOS và Android. Ngoài ra, hệ thống còn có giao diện web dành cho nhà hàng và admin, được xây dựng bằng Angular và PrimeNG, để quản lý thực đơn và theo dõi đơn hàng.

Điểm mạnh của đồ án nằm ở việc thiết kế một kiến trúc phần mềm rõ ràng, dễ kiểm thử và bảo trì, đồng thời giải quyết được bài toán nhiều người cùng chỉnh sửa giỏ hàng mà không bị xung đột nhờ cơ chế kiểm soát đồng thời lạc quan (optimistic concurrency) và đồng bộ real-time. Kết quả thu được là một hệ thống hoàn chỉnh với đầy đủ luồng đặt món cá nhân lẫn đặt món nhóm, đã qua kiểm thử kịch bản và thử nghiệm triển khai thực tế. Tất nhiên, đồ án cũng còn một số hạn chế cần khắc phục trong tương lai như: chưa tích hợp cổng thanh toán thật sự (vẫn đang ở mức demo), chưa có module quản lý tài xế, chưa có tính năng tìm kiếm thông minh dựa trên AI, và chưa có hệ thống quản lý tài chính đầy đủ cho nhà hàng.
\begin{flushright}
Sinh viên thực hiện\\
\begin{tabular}{@{}c@{}}
\textit{(Ký và ghi rõ họ tên)}
\end{tabular}
\end{flushright}

\end{document}
